\documentclass[12pt]{article}

% Load necessary packages
\usepackage{geometry}
\usepackage{graphicx}
\usepackage{hyperref}
\usepackage{listings}
\usepackage{xcolor}
\usepackage[utf8]{inputenc}
\usepackage{hyperref}


% Set page margins
\geometry{left=3cm, right=3cm, top=2cm, bottom=2cm}

% Define colors for hyperlinks
\hypersetup{
    colorlinks=true,
    linkcolor=blue,
    filecolor=blue,      
    urlcolor=blue,
    citecolor=blue
}

\begin{document}

\begin{titlepage}
    \centering
    \includegraphics[width=0.8\textwidth]{image.jpeg}\par % Adjust the width as needed
     \vspace{2cm}
    {\scshape\Large SOEN6841 : Topic Analysis and Synthesis Reprot \par}
    \vspace{1.5cm}
    {\scshape\Huge Distributed Teams Are Founded on Explicit
Communication Channels\par}
    \vspace{1.5cm}
    \vspace{1.5cm}
    {\large Advisor: Professor Pankaj Kamthan\par}
    \vspace{1.5cm}
    {\large By: Chetan Panchal (Student ID: 40227960)\par}
    \vspace{1cm}
    {\large \today\par}
\end{titlepage}

% Table of Contents
\tableofcontents
\newpage

% Abstract Section
\section*{Abstract}

This case study, authored by Juan Pablo Buriticá, emphasizes the critical importance of explicit communication channels in managing distributed teams. As teams grow and spread across various locations, maintaining productivity and efficient management becomes a challenge. The study highlights the necessity of deliberate selection of communication types and channels, particularly in the context of the increasing popularity of remote work. The primary advantage of such explicit communication strategies is enhanced productivity, as it reduces time wasted in seeking information and minimizes unnecessary interruptions.\\
\\The study suggests the use of asynchronous communication channels, like collaborative documents, forums, and emails, for non-urgent information exchange. These channels are effective for situations where real-time interaction is not crucial, thus aiding in avoiding disruptions and facilitating better focus. Conversely, when real-time idea exchange is essential, synchronous channels like group chats, in-person meetings, or video calls are recommended.\\
\\Additionally, the study discusses the dual role of multichannel chats that can function both synchronously and asynchronously but warns against the expectation of constant availability. The case study also addresses the pitfalls of reliance on in-person communication in growing teams, advocating for a culture that prioritizes written communications and explicit communication protocols. This approach not only aligns with the dynamics of remote work but also prepares teams for scalable growth, as exemplified by the use of comprehensive onboarding materials like RFCs and checklists.\\
\\In summary, this case study underlines the significance of explicit communication in distributed teams and presents strategies to optimize productivity, align team members, and prepare for organizational growth.

\newpage

\section{Introduction}

\subsection{Motivation}
\begin{itemize}
  \item The shift towards remote work and the natural growth of companies necessitates effective management of distributed teams.
  \item Challenges in maintaining productivity, alignment, and management efficiency in teams spread across different locations.
  \item An increasing need to understand and implement explicit communication channels to cater to the changing dynamics of team structures and work environments.
  \item The evolution of communication technologies and their impact on team dynamics, especially in a remote or distributed setting.
\end{itemize}

\subsection{Problem Statement}
\begin{itemize}
  \item Investigating the effectiveness of explicit communication channels in distributed teams, focusing on the impact on productivity and team management.
  \item Understanding the balance between asynchronous and synchronous communication and its role in minimizing disruptions and enhancing team focus.
  \item Addressing the challenges of reliance on in-person communication and the transition to more sustainable, scalable communication practices as teams grow.
\end{itemize}

\subsection{Objectives}
\begin{itemize}
  \item To establish clear guidelines and best practices for communication in distributed teams, enhancing overall productivity and team cohesion.
  \item To provide a framework for choosing appropriate communication channels (asynchronous vs. synchronous) based on the team's needs and work nature.
  \item To benefit team leaders, managers, and members by providing insights into effective communication strategies, facilitating better management and faster growth.
  \item To prepare teams for scalable growth and change by advocating for a culture that prioritizes explicit communication and written mediums.
\end{itemize}

\section{Background Material}

\subsection{Subject 1: Communication Challenges in Global Virtual Teams}
\begin{itemize}
  \item Global Virtual Teams (GVTs) face unique challenges due to cultural diversity, geographical dispersion, and reliance on electronic communication. The study by Daim et al. \cite{ref7} highlights areas contributing to communication breakdown in GVTs, including trust, interpersonal relations, cultural differences, leadership, and technology.
  \item The research by Fischer and Mosier \cite{ref2} examines the impact of communication delay and medium on distributed team performance, emphasizing the importance of synchronous versus asynchronous communication.
  \item Shachaf's study \cite{ref8} explores the effects of cultural diversity and information and communication technology on global virtual teams, indicating both positive and negative influences on team effectiveness.
\end{itemize}

\subsection{Subject 2: Effective Communication Strategies in Distributed Teams}
\begin{itemize}
  \item The work by Al-Ani and Edwards \cite{ref4} compares communication in distributed and collocated development teams, providing insights into the evolution of communication models in large-scale software projects.
  \item Lockwood's study \cite{ref6} on virtual team management investigates the causes of communication breakdown in teams, highlighting the rapid organizational changes and the growing trend of offshore work.
\end{itemize}

\subsection{Subject 3: Onboarding and Collaboration in Distributed Software Development}
\begin{itemize}
  \item The study by Britto et al. \cite{ref7} discusses the challenges and strategies for onboarding software developers and teams in globally distributed legacy projects, offering practical insights for managing distributed teams.
  \item Sangwan et al. \cite{ref3} focus on asynchronous collaboration in global software development projects, particularly examining the role of cognitive distance in team collaboration and communication.
\end{itemize}

\section{Methods \& Methodology}

This section outlines the approaches used to investigate the problem of communication in distributed teams and the techniques employed for analyzing the results.

\subsection{Approach 1: Investigating Communication Breakdown}
\begin{itemize}
  \item Method: Conducted a series of interviews and surveys with members of global virtual teams, focusing on identifying the key factors leading to communication breakdown.
  \item Analysis: Utilized Decision Models to analyze the collected data, particularly focusing on trust, interpersonal relations, cultural differences, leadership, and technology \cite{ref1}.
  \item Technique: Employed quantitative methods to validate the impact of identified factors on communication efficiency and team performance.
\end{itemize}

\subsection{Approach 2: Examining the Effect of Communication Delay and Medium}
\begin{itemize}
  \item Method: Conducted an experimental study with teams assigned roles of space crewmembers or flight controllers, collaborating on computer-based tasks with varying communication delays and mediums \cite{ref2}.
  \item Analysis: Analyzed performance variables including time to repair system failures and the number of incorrect repairs, alongside transcriptions of voice communications and logs of chats.
  \item Technique: Applied statistical analysis to assess the impact of synchronous versus asynchronous communication and the effectiveness of different communication mediums under time-delayed conditions.
\end{itemize}

\subsection{Approach 3: Assessing Asynchronous Collaboration in Software Development}
\begin{itemize}
  \item Method: Investigated asynchronous collaboration in global software development projects by examining project artifacts and email communication \cite{ref3}.
  \item Analysis: Performed a sentiment analysis on email content, established communication networks, and analyzed the volume of communication.
  \item Technique: Utilized qualitative analysis to understand the effect of cognitive distance on team sentiment/emotion and collaboration effectiveness.
\end{itemize}

\section{Results Obtained}

This section discusses the results obtained from the methodologies employed in the study, focusing on the conditions under which the results were obtained, the constraints encountered, and the quality of the results.

\subsection{Conditions}
\begin{itemize}
  \item The studies were conducted under varying conditions: virtual team environments, delayed communication scenarios, and asynchronous collaboration settings in software development.
  \item In virtual teams, conditions included cultural diversity, geographical dispersion, and reliance on electronic communication \cite{ref1, ref8}.
  \item Communication delay experiments were conducted under simulated space mission conditions with varying time delays and communication mediums \cite{ref2}.
  \item Asynchronous collaboration was studied within the context of global software development projects, focusing on cognitive distance and its impact \cite{ref3}.
\end{itemize}

\subsection{Constraints}
\begin{itemize}
  \item Constraints included the limitations inherent in electronic communication, such as lack of non-verbal cues and potential for misinterpretation.
  \item Communication delays introduced challenges in real-time decision-making and coordination \cite{ref2}.
  \item Diverse cultural backgrounds and cognitive styles posed challenges in establishing a shared understanding and effective collaboration \cite{ref1, ref3}.
  \item Limitations in the available technology and tools for facilitating virtual communication also acted as constraints.
\end{itemize}

\subsection{Quality of Results}
\begin{itemize}
  \item The quality of results varied based on the study and the specific conditions. In general, results were adequate in highlighting the critical aspects of communication in distributed teams.
  \item Studies on communication breakdown in virtual teams provided significant insights but may not encompass all industries or team dynamics \cite{ref1}.
  \item The experiments on communication delay offered conclusive evidence about the impact of delay and medium, but their applicability might be limited to similar high-stakes, time-sensitive environments \cite{ref2}.
  \item Results from the study on asynchronous collaboration in software development were robust in understanding the role of cognitive distance, offering substantial implications for global software projects \cite{ref3}.
\end{itemize}

\section{Conclusions and Future Works}

This section outlines the conclusions drawn from the study and proposes future directions for research and improvements.

\subsection{Suggested Improvements}
\begin{itemize}
  \item Enhance the diversity of industries and team compositions in future studies to generalize the findings on communication in distributed teams more broadly.
  \item Incorporate more real-time collaborative tools and technologies in experiments to reflect the evolving landscape of remote work.
  \item Future research should also consider the impact of emerging technologies like AI and machine learning on team communication and collaboration.
\end{itemize}

\subsection{Limitations to Solution}
\begin{itemize}
  \item The solutions and findings may not be fully applicable in environments where face-to-face interaction is crucial or in settings that require high levels of spontaneous communication.
  \item Cultural and cognitive diversity, while beneficial in many aspects, can pose challenges in creating a unified team dynamic, limiting the effectiveness of certain communication strategies.
  \item The reliance on technology for communication can create barriers in teams not well-versed with digital collaboration tools.
\end{itemize}

\subsection{Applications in Real World}
\begin{itemize}
  \item The findings can be immediately applied in global software development projects, where teams are geographically dispersed and rely on asynchronous communication.
  \item In multinational companies where teams work across different time zones, the strategies for managing communication delays and choosing the right medium can be highly beneficial.
  \item The insights into virtual team management can aid organizations transitioning to remote work or managing hybrid work environments.
\end{itemize}

\subsection{Conclusion}
In summary, this study provides valuable insights into the challenges and strategies of communication in distributed teams. While effective in many scenarios, these strategies have limitations and need to be adapted to specific team dynamics and project requirements. The findings offer a roadmap for future research and practical application in an increasingly digital and globally distributed work environment.


\section*{References}
\addcontentsline{toc}{section}{References}
\vspace*{-35pt}
\renewcommand{\refname}{}

\begin{thebibliography}{9}
\bibitem{ref1}
"Exploring the communication breakdown in global virtual teams," \emph{International Journal of Project Management}, [Online]. Available: \url{https://www.sciencedirect.com/science/article/abs/pii/S0263786311000779}.

\bibitem{ref2}
"The Impact of Communication Delay and Medium on Team Performance and Communication in Distributed Teams," \emph{Proceedings of the Human Factors and Ergonomics Society Annual Meeting}, [Online]. Available: \url{https://journals.sagepub.com/doi/abs/10.1177/1541931214581025}.

\bibitem{ref3}
"Asynchronous Collaboration: Bridging the Cognitive Distance in Global Software Development Projects," \emph{IEEE}, [Online]. Available: \url{https://ieeexplore.ieee.org/abstract/document/9247281}.

\bibitem{ref4}
"A Comparative Empirical Study of Communication in Distributed and Collocated Development Teams," \emph{IEEE}, [Online]. Available: \url{https://ieeexplore.ieee.org/abstract/document/4638651}.

\bibitem{ref5}
"A study of the effect of communication channel and authority on group decision processes and outcomes," \emph{Decision Support Systems}, [Online]. Available: \url{https://www.sciencedirect.com/science/article/abs/pii/S0167923698000487}.

\bibitem{ref6}
"Virtual team management: what is causing communication breakdown," in \emph{Language and Intercultural Communication in the Workplace}, J. Lockwood, Ed. Routledge, [Online]. Available: \url{https://www.taylorfrancis.com/chapters/edit/10.4324/9781315468174-15/virtual-team-management-causing-communication-breakdown-jane-lockwood}.

\bibitem{ref7}
"Onboarding software developers and teams in three globally distributed legacy projects," \emph{Software: Practice and Experience}, [Online]. Available: \url{https://onlinelibrary.wiley.com/doi/abs/10.1002/smr.1921}.

\bibitem{ref8}
"Cultural diversity and information and communication technology impacts on global virtual teams," \emph{Information \& Management}, [Online]. Available: \url{https://www.sciencedirect.com/science/article/abs/pii/S0378720608000153}.

\bibitem{ref9}
"Multi-channel Data Acquisition and Wireless Communication FPGA-Based System, to Real-Time Remote Monitoring," \emph{IEEE}, [Online]. Available: \url{https://ieeexplore.ieee.org/abstract/document/8241338}.
\end{thebibliography}

\end{document}
